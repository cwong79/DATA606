\documentclass[]{article}
\usepackage{lmodern}
\usepackage{amssymb,amsmath}
\usepackage{ifxetex,ifluatex}
\usepackage{fixltx2e} % provides \textsubscript
\ifnum 0\ifxetex 1\fi\ifluatex 1\fi=0 % if pdftex
  \usepackage[T1]{fontenc}
  \usepackage[utf8]{inputenc}
\else % if luatex or xelatex
  \ifxetex
    \usepackage{mathspec}
  \else
    \usepackage{fontspec}
  \fi
  \defaultfontfeatures{Ligatures=TeX,Scale=MatchLowercase}
\fi
% use upquote if available, for straight quotes in verbatim environments
\IfFileExists{upquote.sty}{\usepackage{upquote}}{}
% use microtype if available
\IfFileExists{microtype.sty}{%
\usepackage{microtype}
\UseMicrotypeSet[protrusion]{basicmath} % disable protrusion for tt fonts
}{}
\usepackage[margin=1in]{geometry}
\usepackage{hyperref}
\hypersetup{unicode=true,
            pdftitle={Homework 2},
            pdfauthor={Calvin Wong},
            pdfborder={0 0 0},
            breaklinks=true}
\urlstyle{same}  % don't use monospace font for urls
\usepackage{graphicx,grffile}
\makeatletter
\def\maxwidth{\ifdim\Gin@nat@width>\linewidth\linewidth\else\Gin@nat@width\fi}
\def\maxheight{\ifdim\Gin@nat@height>\textheight\textheight\else\Gin@nat@height\fi}
\makeatother
% Scale images if necessary, so that they will not overflow the page
% margins by default, and it is still possible to overwrite the defaults
% using explicit options in \includegraphics[width, height, ...]{}
\setkeys{Gin}{width=\maxwidth,height=\maxheight,keepaspectratio}
\IfFileExists{parskip.sty}{%
\usepackage{parskip}
}{% else
\setlength{\parindent}{0pt}
\setlength{\parskip}{6pt plus 2pt minus 1pt}
}
\setlength{\emergencystretch}{3em}  % prevent overfull lines
\providecommand{\tightlist}{%
  \setlength{\itemsep}{0pt}\setlength{\parskip}{0pt}}
\setcounter{secnumdepth}{0}
% Redefines (sub)paragraphs to behave more like sections
\ifx\paragraph\undefined\else
\let\oldparagraph\paragraph
\renewcommand{\paragraph}[1]{\oldparagraph{#1}\mbox{}}
\fi
\ifx\subparagraph\undefined\else
\let\oldsubparagraph\subparagraph
\renewcommand{\subparagraph}[1]{\oldsubparagraph{#1}\mbox{}}
\fi

%%% Use protect on footnotes to avoid problems with footnotes in titles
\let\rmarkdownfootnote\footnote%
\def\footnote{\protect\rmarkdownfootnote}

%%% Change title format to be more compact
\usepackage{titling}

% Create subtitle command for use in maketitle
\newcommand{\subtitle}[1]{
  \posttitle{
    \begin{center}\large#1\end{center}
    }
}

\setlength{\droptitle}{-2em}

  \title{Homework 2}
    \pretitle{\vspace{\droptitle}\centering\huge}
  \posttitle{\par}
    \author{Calvin Wong}
    \preauthor{\centering\large\emph}
  \postauthor{\par}
      \predate{\centering\large\emph}
  \postdate{\par}
    \date{9/8/2018}


\begin{document}
\maketitle

2.6 Dice rolls. If you roll a pair of fair dice, what is the probability
of (a) getting a sum of 1? Getting a sum of 1 with a pair of dice is
impossible, therefore, the probability is 0. (b) getting a sum of 5?
There are 4 combinations; (1,4), (2,3), (3,2), (4,1). Since the total
combinations is 36. The probability of getting a sum of 5 is 4/36 =
0.111 (c) getting a sum of 12? There is 1 combination; (6,6). Since the
total combinations is 36. The probability of getting a sum of 12 is 1/36
= 0.027

2.8 Poverty and language. The American Community Survey is an ongoing
survey that provides data every year to give communities the current
information they need to plan investments and services. The 2010
American Community Survey estimates that 14.6\% of Americans live below
the poverty line, 20.7\% speak a language other than English (foreign
language) at home, and 4.2\% fall into both categories. (a) Are living
below the poverty line and speaking a foreign language at home disjoint?
No, because the text refers that living below the poverty line and
speaking a foreign language falls into both categories. (b) Draw a Venn
diagram summarizing the variables and their associated probabilities.
(c) What percent of Americans live below the poverty line and only speak
English at home? 14.6\% - 4.2\% = 10.4\% (d) What percent of Americans
live below the poverty line or speak a foreign language at home? (20.7\%
+ 14.6\%) - 4.2\% = 31.1\% (e) What percent of Americans live above the
poverty line and only speak English at home? 100\% - (14.6\% + 20.7\% -
4.2\%) = 68.9\% (f) Is the event that someone lives below the poverty
line independent of the event that the person speaks a foreign language
at home? Is P(Poverty) x P(Foreign Language) = P(Poverty and Foreign
Language), 0.146 x 0.207 = 0.042? 0.030 does not equal 0.042. Therefore,
the events are not independent of each other.

2.20 Assortative mating. Assortative mating is a nonrandom mating
pattern where individuals with similar genotypes and/or phenotypes mate
with one another more frequently than what would be expected under a
random mating pattern. Researchers studying this topic collected data on
eye colors of 204 Scandinavian men and their female partners. The table
below summarizes the results. For simplicity, we only include
heterosexual relationships in this exercise. (a) What is the probability
that a randomly chosen male respondent or his partner has blue eyes?
P(Sm Blue or Pf Blue) = (P(Sm Blue) + P(Pf Blue) - P(Sm Blue and Pf
Blue)) / P(Ttl) = (108 + 114 - 78) / 204 = 0.7058 \textasciitilde{}
70.58\% (b) What is the probability that a randomly chosen male
respondent with blue eyes has a partner with blue eyes? P(Sm Blue and Pf
Blue) = (P(Sm Blue) / P(Ttl)) / (P(Sm Blue Ttl) / P(Ttl)) = (78 / 204) /
(114 / 204) = 0.6842 \textasciitilde{} 68.42\% (c) What is the
probability that a randomly chosen male respondent with brown eyes has a
partner with blue eyes? P(Sm Brown and Pf Blue) = (P(Sm Brown) / P(Ttl))
/ (P(Sm Brown Ttl) / P(Ttl)) = (19 / 204) / (54 / 204) = 0.3518
\textasciitilde{} 35.18\% What about the probability of a randomly
chosen male respondent with green eyes having a partner with blue eyes?
P(Sm Green and Pf Blue) = (P(Sm Green) / P(Ttl)) / (P(Sm Green Ttl) /
P(Ttl)) = (11 / 204) / (36 / 204) = 0.3055 \textasciitilde{} 30.55\% (d)
Does it appear that the eye colors of male respondents and their
partners are independent? Explain your reasoning.

2.30 Books on a bookshelf. The table below shows the distribution of
books on a bookcase based on whether they are nonfiction or fiction and
hardcover or paperback. (a) Find the probability of drawing a hardcover
book first then a paperback fiction book second when drawing without
replacement. P(Hardcover then Paperback Fiction) = P(Hardcover) *
P(Paperback Fiction) = (28/95) * (54/94) = 0.1693 \textasciitilde{}
16.93\% (b) Determine the probability of drawing a fiction book first
and then a hardcover book second, when drawing without replacement.
P(Fiction then Hardcover) = P(Fiction) * P(Hardcover) = (72/95) *
(28/94) = 0.2257 \textasciitilde{} 22.57\% (c) Calculate the probability
of the scenario in part (b), except this time complete the calculations
under the scenario where the first book is placed back on the bookcase
before randomly drawing the second book. P(Fiction then Hardcover) =
P(Fiction) * P(Hardcover) = (72/95) * (28/95) = 0.2233 \textasciitilde{}
22.33\% (d) The final answers to parts (b) and (c) are very similar.
Explain why this is the case. The denominator of P(Hardcover) changed
from 94 to 95 due to replacement. It is too small of a number to make a
significant difference.

2.38 Baggage fees. An airline charges the following baggage fees: \$25
for the first bag and \$35 for the second. Suppose 54\% of passengers
have no checked luggage, 34\% have one piece of checked luggage and 12\%
have two pieces. We suppose a negligible portion of people check more
than two bags. (a) Build a probability model, compute the average
revenue per passenger, and compute the corresponding standard deviation.
x0 = 0, x1 = \$25, x2 = \$35 p0 = .54, p1 = .34, p2 = .12 E(X) = (0 *
.54) + (\$25 * .34) + (\$60 * .12) = \$15.70 sd(X) = (((0 - 15.70)\^{}2
* .54) + ((25 - 15.70)\^{}2 * .34) + ((60 - 15.70)\^{}2 * .12))\^{}1/2 =
(133.10 + 29.40 + 235.49)\^{}1/2 = (397.99)\^{}1/2 = 19.94 (b) About how
much revenue should the airline expect for a flight of 120 passengers?
With what standard deviation? Note any assumptions you make and if you
think they are justified. E(120) = \$15.70 * 120 = \$1884 sd(120) = (((0
- 15.70)\^{}2 * .54 * 120) + ((25 - 15.70)\^{}2 * .34 * 120) + ((60 -
15.70)\^{}2 * .12 * 120))\^{}1/2 = (15972 + 3528 + 28248.8)\^{}1/2 =
\$218.53

2.44 Income and gender. The relative frequency table below displays the
distribution of annual total personal income (in 2009 inflation-adjusted
dollars) for a representative sample of 96,420,486 Americans. These data
come from the American Community Survey for 2005-2009. This sample is
comprised of 59\% males and 41\% females.69 (a) Describe the
distribution of total personal income. The distribution is pretty even
with a slight skew to the right. It peaks at salary range from \$35k -
\$49.9k. (b) What is the probability that a randomly chosen US resident
makes less than \$50,000 per year? P(\textless{}\$50k) = (2.2\% + 4.7\%
+ 15.8\% + 18.3\% + 21.2\%) = 62.2\% (c) What is the probability that a
randomly chosen US resident makes less than \$50,000 per year and is
female? Note any assumptions you make. P(\textless{}\$50k and female) =
62.2\% * 41\% = 25.50\% *Assuming females are normally distributed
between groups (d) The same data source indicates that 71.8\% of females
make less than \$50,000 per year. Use this value to determine whether or
not the assumption you made in part (c) is valid. The assumption I made
is invalid. 71.8\% of females make less than \$50k indicates that the
data is not normally distributed as assumed in part (c). When taking
this information into account, for females, the income for females is in
fact left skewed.


\end{document}
